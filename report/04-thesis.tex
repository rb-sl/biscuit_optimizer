% Optimization Project: Biscuit Optimizer
% Roberto Basla
% Politecnico di Milano
% A.Y. 2021/2022

\section{Further applications: histological images augmentation}
This section shows a possible application of the devised nesting heuristics to the problem of data augmentation.

\subsection{The problem}
Data augmentation is a series of techniques used to increase the amount of data by adding slightly modified copies of already existing or newly created synthetic data.
the biscuit optimization problem can be applied to the problem of data augmentation, i.e. the generation of realistic synthetic data to improve the training of machine learning systems.

In particular, previous heuristics will be applied to image augmentation in biomedical images which benefit most from this process as obtaining samples can be hard or very expensive.

Images are taken from public datasets available at [https://www.kaggle.com/competitions/data-science-bowl-2018/data](https://www.kaggle.com/competitions/data-science-bowl-2018/data)

\subsection{Heuristic examples}



